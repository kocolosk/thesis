\section{$A_{LL}$ Methodology}

Let's start with the equation for a raw double spin asymmetry:

\begin{equation}
  A_{LL} = \frac{\sum_{runs} P_{Y}P_{B}(N_{++} - RN_{+-})}{\sum_{runs} P_{Y}^{2}P_{B}^{2}(N_{++} + RN_{+-})}
\end{equation}
%
where R is a ratio of spin-dependent luminosities determined from the BBC scaler system:

\begin{equation}
  R = \frac{\mathcal{L}_{UU}+\mathcal{L}_{DD}}{\mathcal{L}_{DU}+\mathcal{L}_{UD}}
\end{equation}
%
Each of the $\mathcal{L}_{ij}$ is a sum of the scaler counts from board 5 for
timebins 7, 8, and 9 for events with a given spin state (in terms of 4 bit
spin states, UU = 5, DU = 6, UD = 9, and DD = 10). The formula for the
statistical uncertainty on $A_{LL}$ neglects uncertainties on the relative
luminosities and beam polarizations. Assuming Poisson statistics on $N_{++}$
and $N_{+-}$ we have for a single run

\begin{equation}
  \left(\frac{\sigma_{A_{LL}}}{A_{LL}}\right)^2 = \frac{N_{++} + R^2 N_{+-}}{(N_{++} - R N_{+-})^2} + \frac{N_{++} + R^2 N_{+-}}{(N_{++} + R N_{+-})^2} - 2\frac{N_{++} + R^2 N_{+-}}{N_{++}^2 - R^2 N_{+-}^2} \times COV(N_{++} - R N_{+-}, N_{++} + R N_{+-})
\end{equation}
%
where the covariance term is just

\begin{equation}
  COV(N_{++} - R N_{+-},~N_{++} + R N_{+-}) = N_{++} - R^2 N_{+-}
\end{equation}
%
In the case of small asymmetries, the relative uncertainty on the numerator
dominates the uncertainty on $A_{LL}$:

\begin{eqnarray}
  \left(\frac{\sigma_{A_{LL}}}{A_{LL}}\right)^2 & = & \left(N_{++} + R^2 N_{+-}\right) \left[\frac{1}{(N_{++} - R N_{+-})^2} + \frac{1}{(N_{++} + R N_{+-})^2} - \frac{2(N_{++}-R^2 N_{+-})}{N_{++}^2 - R^2 N_{+-}^2} \right] \\
  & \approx & \frac{N_{++} + R^2 N_{+-}}{(N_{++} - R N_{+-})^2}
\end{eqnarray}
%
ROOT's TH1::Divide method does the error propagation correctly in this limit
(it ignores the covariance), so we rely on it to calculate the results. The
generalization to a sum over runs is straightforward since the yields for each
run are uncorrelated.

\subsection{Multi-Particle Statistics}

In this analysis it's very often the case that we accept multiple pions from a
single event. Treating each of these particles as an independent event and
simply using $\sqrt{N}$ for the errors as we did in the previous section is
not quite correct. Following the prescription in
\href{http://www.star.bnl.gov/protected/spin/sowinski/analysis/derivations/multiParticle.pdf}{Jim's
note on multi-particle statistics} we fill each bin in a histogram at most
once per event, using a weight equal to the number of particles that fell into
that bin. Note that this approach does not account for correlations across
bins.

\subsection{Background Subtraction}

Next we consider asymmetries with sideband subtraction. Specifically, the raw
charged pion asymmetries in this analysis are contaminated by protons, kaons,
and electrons. The protons and kaons are necessarily combined into a single
sideband. Both sidebands have some non-negligible pion contribution. We start
by defining a reduced background fraction that accounts for the impurities in
the sideband:

\begin{equation}
  f_{x}(y) = \frac{x~counts~in~y~window}{total~in~y~window}
\end{equation}

\begin{equation}
  f'(x) = \frac{f_{x}(\pi)}{1 - f_{\pi}(x)}
\end{equation}
%
The standard equations for the background-subtracted $A_{LL}^{\pi}$ and its
statistical uncertainty are only modified by replacing the background fraction
for each sideband with its reduced background fraction:

\begin{equation}
  A_{LL}^{\pi} = \frac{ A_{LL}^{\pi,raw} - f'(p+K)A_{LL}^{p+K,raw} - f'(e)A_{LL}^{e,raw} }{1 - f'(p+K) - f'(e) }
  \label{eqn:all}
\end{equation}

\begin{equation}
  \sigma_{A_{LL}^{\pi}} = \frac{\sqrt{ \sigma_{A_{LL}^{raw}}^{2} + f'(p+K)^{2} * \sigma_{A_{LL}^{p+K,raw}}^{2} + f'(e)^{2} * \sigma_{A_{LL}^{e,raw}}^{2} }}{1 - f'(p+K) - f'(e)}
  \label{eqn:sigma-all}
\end{equation}
