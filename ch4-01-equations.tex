\section{$A_{LL}$ Methodology}

We begin with the equation for a double spin asymmetry in terms of directly measurable quantities,
%
\begin{equation}
  A_{LL} = \frac{\sum_{runs} P_{Y}P_{B}\left[(N_{uu} + N_{dd}) - R(N_{ud} + N_{du})\right]}{\sum_{runs} P_{Y}^{2}P_{B}^{2}\left[(N_{uu} + N_{dd}) + R(N_{ud} + N_{du})\right]}.
  \label{eqn:all-basics}
\end{equation}
%
Henceforth parity conservation is employed; the \(++\) subscript denotes a sum
over the \(uu\) and \(dd\) states and the \(+-\) subscript a sum over \(ud\)
and \(du\) states. \(P_Y\) and \(P_B\) are the Yellow and Blue RHIC beam
polarizations, \(N_{ij}\) are spin-dependent charged pion yields, and \(R =
\frac{\mathcal{L}_{++}}{\mathcal{L}_{+-}}\) is a ratio of sampled luminosities
in different spin configurations. Each of the $\mathcal{L}_{ij}$ is a sum of
the scaler counts over bunch crossings with that spin state.

The formula for the statistical uncertainty on \(A_{LL}\) neglects
uncertainties on the relative luminosities and beam polarizations. Assuming
Poisson statistics on \(N_{++}\) and \(N_{+-}\) the uncertainty on the
asymmetry for a single run is
%
\begin{equation}
  \left(\frac{\sigma_{A_{LL}}}{A_{LL}}\right)^2 = \left(N_{++} + R^2 N_{+-}\right)\left[\frac{1}{N^2} + \frac{1}{D^2} - \frac{2 \times COV(N,D)}{ND} \right]
\end{equation}
%
where \(N\) and \(D\) are the numerator and denominator of the raw asymmetry
(that is, neglecting polarizations), and \(COV(N,D) = N_{++} - R^2 N_{+-}\).
In the case of small asymmetries \(\frac{1}{N^2} \gg (\frac{1}{D^2} - \frac{2
COV(N,D)}{ND}\)), and the uncertainty on the numerator dominates the uncertainty
on \(A_{LL}\):
%
\begin{equation}
  \left(\frac{\sigma_{A_{LL}}}{A_{LL}}\right)^2 \approx \frac{N_{++} + R^2 N_{+-}}{N^2}
\end{equation}
%
ROOT's TH1::Divide method does the error propagation correctly in this limit
(it ignores the covariance term). The generalization to a sum over runs is
straightforward since the yields for each run are uncorrelated.

\subsection{Multi-Particle Statistics}

This analysis often accepts multiple pions from a single event. Treating each
of the particles as an independent event and na\"ively using
$\sqrt{N_{pions}}$ for the statistical uncertainty as in the previous section
is not correct. Following the prescription in
\cite{sowinski-multiparticle-statistics} each bin in a histogram is
incremented at most once per event, using a weight equal to the number of
particles that fell into the bin. This technique neglects intra-event particle
correlations across histogram bins.

\subsection{Background Subtraction}

As discussed in Section \ref{sec:pid}, a particle's energy loss per unit path
length (dE/dx) in the TPC provides an effective means of identification across
a broad range of momenta. In the momentum range of interest (2.0 GeV/c and up)
charged pions are not fully separated from heavier charged hadrons (primarily protons and kaons, which have a smaller \(\langle dE/dx \rangle\)) and
electrons (which have a larger \(\langle dE/dx \rangle\)). This contamination is addressed by subtracting the asymmetry of
the background from the raw asymmetry calculated using all
particles in the charged pion acceptance window. The background sidebands located on both sides of the acceptance window have different physical sources and potentially different double spin asymmetries.  The subtraction
procedure must also account for the lack of a clean background sample; the
sidebands themselves have a non-negligible contamination from the charged pion
signal. We start by defining a reduced background fraction to account for the
impurities in the sideband:

\begin{equation*}
  f_{x}(y) = \frac{x~\mathrm{counts~in}~y~\mathrm{window}}{\mathrm{total~ in}~y~\mathrm{window}}
\end{equation*}

\begin{equation*}
  f'(x) = \frac{f_{x}(\pi)}{1 - f_{\pi}(x)}
\end{equation*}
%
The standard equations for the background-subtracted $A_{LL}^{\pi}$ and its
statistical uncertainty are only modified by replacing the background fraction
for each sideband with its reduced background fraction:

\begin{equation}
  A_{LL}^{\pi} = \frac{ A_{LL}^{\pi,raw} - f'(p+K)A_{LL}^{p+K,raw} - f'(e)A_{LL}^{e,raw} }{1 - f'(p+K) - f'(e)},
  \label{eqn:all}
\end{equation}

\begin{equation}
  \sigma_{A_{LL}^{\pi}} = \frac{\sqrt{ \sigma_{A_{LL}^{raw}}^{2} + f'(p+K)^{2} * \sigma_{A_{LL}^{p+K,raw}}^{2} + f'(e)^{2} * \sigma_{A_{LL}^{e,raw}}^{2} }}{1 - f'(p+K) - f'(e)}.
  \label{eqn:sigma-all}
\end{equation}
%
These equations are the final formulae used to calculate the asymmetries for a single STAR run in this work.  The \(raw\) superscript in the formulae denotes an \(A_{LL}\) calculated without background subtraction but taking multi-particle statistics into account.

Equation \ref{eqn:all-basics} specified beam polarizations, spin-dependent relative luminosities, and spin-sorted particle yields as the three components to a double spin asymmetry measurement.  A discussion of the extraction of each of these quantities follows.