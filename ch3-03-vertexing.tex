\section{Vertexing}

STAR uses different vertex finders optimized for the very different running
conditions of heavy ion (MinuitVF) and proton-proton (PPV) collisions. A
detailed description of both can be found in STAR Note 488
\cite{vertex-finder-starnote}; the discussion here will focus on the Pile-Up
Proof Vertexer (PPV), but many of the basic principles apply to both
algorithms.

One of PPV's major design goals is robustness in the face of high levels of
pileup. Pileup events usually occur in one of the $\sim$40 other bunch
crossings that are read out along with the trigger bunch crossing by the TPC,
although on rare occasions a single bunch crossing can contain multiple hard
scatterings. PPV does not try to guard against this in-time pileup, but it
does work to suppress false event vertices from other bunch crossings.

PPV relies on a pre-calculated beam line constraint for the $x$ and $y$
position of each vertex. The beam line constraint is determined by running the
Minuit Vertex Finder on a subset of high-multiplicity events and fitting the
resulting distribution of event vertices with a straight line. PPV uses a
subset of TPC tracks satisfying a series of quality cuts; these include a
transverse momentum greater than 200 MeV, a distance of closest approach (DCA)
to the beam line of less than 3 cm, and a requirement that the number of hits
used to fit the track is at least 70\% of the maximum number of possible hits
for the track's helix. Each track is given an initial weight based on
projection of its DCA in the $xy$ plane and the errors on the extrapolation to
the beam line used to calculate that DCA. The track's weight is increased if
it is reconstructed from a minimum number of hits on both sides of the central
membrane of the TPC or is matched to an energy deposit in the BEMC or EEMC,
since tracks satisfying one or more of those conditions are very likely to
have been produced in the bunch crossing that fired the trigger(s). A track
with one of these positive weight adjustments is called a ``matched'' track.
Conversely, if a track extrapolates to an active calorimeter tower without any
energy deposit, or crosses the central membrane but has fewer than the minimum
required number of hits on both sides of the membrane, the track's weight is
reduced. The total weight for the track is the product of the initial weight
and each of these adjustments.

Vertices are identified by binning the $z$ axis with 1mm resolution and
generating a histogram of all the track weights. The peak of the histogram is
the first vertex candidate; all tracks which extrapolate to within 3cm of this
vertex position are grouped with the candidate and removed from further
analysis. PPV iterates this procedure until no more vertex candidates can be
found. In the 2005 and 2006 runs PPV required at least two ``matched'' tracks
in order to save a vertex; that requirement has been relaxed in recent years
to improve the vertex finding efficiency for forward triggers, which have
event topologies that make it unlikely to find two tracks in an event
satisfying the matching conditions. PPV saves multiple vertices for each
event, ordered by rank; in 2005 and 2006, the rank is simply the cumulative
weight of all the tracks in the vertex. The vertex finding efficiency is
highly trigger-dependent; Table \ref{tbl:vertex-finding-efficiencies} lists
the fraction of events with at least one primary vertex in each of the
triggers used for this analysis.

\begin{table}
  \begin{center}
    \begin{tabular}{cc|rrr}
      \hline
      year & trigger ID & total events & events with vertex & efficiency\\
      \hline
      \hline
      2005 & MB (96011) & 1,686,762 & 1,084,203 & 64.3\%\\
      \hline
      2005 & BJP1 (96221) & 1,878,465 & 1,835,016 & 97.7\%\\
      \hline
      2005 & BJP2 (96233) & 5,446,354 & 5,243,167 & 96.3\%\\
      \hline
      2006 & MB (117001) & 257,291 & 131,657 & 51.2\%\\
      \hline
      2006 & BJP1 (13722[1-2]) & 3,294,257 & 3,138,997 & 95.3\%\\
      \hline
    \end{tabular}
  \end{center}
  \caption{Vertex Finding Efficiencies for events analyzed in this work.}
  \label{tbl:vertex-finding-efficiencies}
\end{table}

Once the vertices have been identified, the tracker adds the vertex position
as an additional hit to every track associated with that vertex and refits
those tracks. The STAR software framework saves this collection of ``primary''
tracks as well as the original ``global'' tracks. In this analysis primary
tracks are used exclusively; the global tracks associated with those primaries
are only analyzed to apply a more stringent cut on the DCA to the primary
vertex position obtained from the original track extrapolation.

\begin{figure}
  \includegraphics[width=1.0\textwidth]{figures/ppv-candidate-distribution}
  \caption{Example vertex candidate distribution generated by PPV
  \cite{vertex-finder-starnote}. An ordered list of vertices is extracted from
  the peaks of this distribution, with the requirement that each vertex
  contains at least two tracks satisfying the matching conditions.}
  \label{fig:ppv-candidate-distribution}
\end{figure}
