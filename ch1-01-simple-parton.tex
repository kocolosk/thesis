\section{The Simple Parton Model}

% predates QCD, invented to explain scaling of F_{1,2}
% Leader cites 69 Feynman PRL as inventor of parton model, but I skimmed the Letter and I don't see it.
% \cite{Panofsky:1968pb} -- confirmation of Bjorken scaling
% \cite{Bjorken:1968dy} -- Bjorken scaling

In the late nineteen-sixties high energy physicists at SLAC confirmed
Bjorken's hypothesis that the inelastic structure functions of the proton
scaled; that is, at high energies they did not depend on the $Q^2$ of the
interaction. This result stood in sharp contrast to the power-law behavior of
the proton's elastic form factors, and implied the existence of point-like
constituents inside the proton. These ``partons'' were conceived as
effectively massless, electromagnetically charged particles; in the
deep-inelastic scattering (DIS) regime, a virtual photon interacts with a
parton, not the proton as a whole.

Scaling is manifest when Bjorken's $F_1$ structure function is expressed in
terms of the number densities $q(x)$ of quarks and $\bar q(x)$ of antiquarks
as
%
\begin{equation}
  F_1(x, Q^2) = \frac{1}{2}\sum_{j}{e_j^2[q_j(x) + \bar{q}_j(x)]}
\end{equation}
%
where the sum is taken over quark flavors $j$ and $e_j$ is the electromagnetic
charge of flavor $j$. In longitudinally polarized DIS we define an analogous
polarization density $\Delta q(x) \equiv q_+(x) - q_-(x)$ as the difference in
number density between quarks whose spins are aligned with the (longitudinal)
spin of the proton and quarks whose spins are anti-aligned; the polarized
analogue to $F_1$ is then
%
\begin{equation}
  g_1(x, Q^2) = \frac{1}{2}\sum_{j}{e_j^2[\Delta q_j(x) + \Delta \bar{q}_j(x)]}.
\end{equation}

In the na\"ive parton model one assumes \(SU(3)_F\) flavor symmetry and thus it is useful to express the integral of \(g_1\) in terms of quantities which have specific \(SU(3)_F\) transformation properties:
%
\begin{equation}
  \Gamma_1^p = \int_0^1 dx~g_1^p(x) = \frac{1}{9}\left[a_0 + \frac{3}{4}a_3 + \frac{1}{4}a_8\right].
  \label{eqn:g1}
\end{equation}
%
The \(a_j\) are the hadronic matrix elements of an octet of quark $SU(3)_F$
axial-vector currents $J_{5\mu}^i$ and a flavor singlet axial current
$J_{5\mu}^0$, and are related to the polarized quark densities in the proton as
%
\begin{eqnarray}
  a_0 & = & (\Delta u + \Delta \bar{u}) + (\Delta d + \Delta \bar{d}) + (\Delta s + \Delta \bar{s}) \nonumber \\
  a_3 & = & (\Delta u + \Delta \bar{u}) - (\Delta d + \Delta \bar{d}) \nonumber \\
  a_8 & = & (\Delta u + \Delta \bar{u}) + (\Delta d + \Delta \bar{d}) - 2(\Delta s + \Delta \bar{s})
  \label{eqn:su3-dis}
\end{eqnarray}
%
In the limit of massless partons the non-singlet currents are scale-independent quantities, and are known from $\beta$-decay measurements
\cite{}:
%
\begin{eqnarray}
  a_3 & = & g_A = F+D = 1.2670 \pm 0.0035 \nonumber \\
  a_8 & = & 3F-D = 0.585 \pm 0.025.
  \label{eqn:beta-decay}
\end{eqnarray}
%
Hence a measurement of \(\Gamma_1^p\) allows the extraction of the flavor singlet \(a_0\), the quark spin contribution to the spin of the proton.  If one assumes that the sea quark distribution is either unpolarized or
CP-symmetric and thus does not contribute to the spin of the proton, as Ellis and Jaffe did in 1974 \cite{Ellis:1973kp}, Equations \ref{eqn:su3-dis} and \ref{eqn:beta-decay} allow a \textit{prediction} of the quark spin contribution to the spin of the proton, namely \(a_0 = a_8 \approx 0.6\).
