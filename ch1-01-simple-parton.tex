\section{The Simple Parton Model}

Over the past century studies of spin in elementary particle physics have proven their worth time and again, exposing weaknesses in theories that were otherwise able to explain the measurements of the day.  The first indication that the proton was itself a composite particle came from a spin experiment, namely Stern's discovery that the magnetic moment of the proton is incompatible with the Dirac prediction for spin-\(\frac{1}{2}\) particles.  As a result, the \textit{structure function} was introduced in scattering cross sections to codify our lack of knowledge about the true internal structure of the nucleon.

In the nineteen-sixties Gell-Mann and Feynman independently proposed models in which the proton is composed of a set of point-like elementary particles. Bjorken realized that if these ``partons'', to use Feynman's term, were more than a convenient theoretical construct the structure functions of the proton would acquire a scaling behavior at high energies; that is, they would become independent of the momentum transfer of the probe.  In this deep-inelastic scattering (DIS) regime, the probe would interact with a parton, not the proton as a whole.  Kendall, Friedman, and Taylor observed this scaling behavior at SLAC a few years later, and the simple parton model was born.

In the simple parton model Bjorken's $F_1$ structure function is expressed in
terms of the number densities $q(x)$ of quarks and $\bar q(x)$ of antiquarks
as
%
\begin{equation}
  F_1(x, Q^2) = \frac{1}{2}\sum_{j}{e_j^2[q_j(x) + \bar{q}_j(x)]}
\end{equation}
%
where the sum is taken over quark flavors $j$ and $e_j$ is the electromagnetic
charge of flavor $j$. In longitudinally polarized DIS we define an analogous
polarization density $\Delta q(x) \equiv q_+(x) - q_-(x)$ as the difference in
number density between quarks whose spins are aligned with the (longitudinal)
spin of the proton and quarks whose spins are anti-aligned; the polarized
analogue to $F_1$ is then
%
\begin{equation}
  g_1(x, Q^2) = \frac{1}{2}\sum_{j}{e_j^2[\Delta q_j(x) + \Delta \bar{q}_j(x)]}.
  \label{eqn:simple-g1}
\end{equation}

In the na\"ive parton model one assumes \(SU(3)_F\) flavor symmetry and thus it is useful to express the integral of \(g_1\) in terms of quantities which have specific \(SU(3)_F\) transformation properties:
%
\begin{equation}
  \Gamma_1^p = \int_0^1 dx~g_1^p(x) = \frac{1}{9}\left[a_0 + \frac{3}{4}a_3 + \frac{1}{4}a_8\right].
  \label{eqn:g1}
\end{equation}
%
The \(a_j\) are the hadronic matrix elements of an octet of quark $SU(3)_F$
axial-vector currents $J_{5\mu}^i$ and a flavor singlet axial current
$J_{5\mu}^0$, and are related to the polarized quark densities in the proton as
%
\begin{eqnarray}
  a_0 & = & (\Delta u + \Delta \bar{u}) + (\Delta d + \Delta \bar{d}) + (\Delta s + \Delta \bar{s}) \nonumber \\
  a_3 & = & (\Delta u + \Delta \bar{u}) - (\Delta d + \Delta \bar{d}) \nonumber \\
  a_8 & = & (\Delta u + \Delta \bar{u}) + (\Delta d + \Delta \bar{d}) - 2(\Delta s + \Delta \bar{s})
  \label{eqn:su3-dis}
\end{eqnarray}
%
In the limit of massless partons the non-singlet currents are scale-independent quantities, and are known from $\beta$-decay measurements
\cite{Amsler:2008zzb}:
% Stiegler says a_8 = 3F+D, but Anselmino and Ashman agree on the (-)
\begin{eqnarray}
  a_3 & = & g_A = F+D = 1.2670 \pm 0.0035 \nonumber \\
  a_8 & = & 3F-D = 0.585 \pm 0.025.
  \label{eqn:beta-decay}
\end{eqnarray}
%
Hence a measurement of \(\Gamma_1^p\) allows the extraction of the flavor singlet \(a_0\), the quark spin contribution to the spin of the proton.  If one assumes that the sea quark distribution is either unpolarized or
CP-symmetric and thus does not contribute to the spin of the proton, as Ellis and Jaffe did in 1974 \cite{Ellis:1973kp}, Equations \ref{eqn:su3-dis} and \ref{eqn:beta-decay} allow a \textit{prediction} of the quark spin contribution to the spin of the proton, namely \(a_0 = a_8 \approx 0.6\).
