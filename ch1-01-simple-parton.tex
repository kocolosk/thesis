\section{The Simple Parton Model}

Over the past century studies of spin in elementary particle physics have proven
their worth time and again, exposing weaknesses in theories that were otherwise
able to explain the measurements of the day. The first indication that the
proton was itself a composite particle came from a spin experiment, namely
Stern's discovery that the magnetic moment of the proton is incompatible with
the Dirac prediction for spin-\(\frac{1}{2}\) particles. As a result, the
\textit{structure function} was introduced in scattering cross sections to
codify our lack of knowledge about the true internal structure of the nucleon.

In 1964, Gell-Mann and Zweig independently proposed models
\cite{GellMann:1964nj, Zweig:1964jf} in which hadrons are composed of a set of
point-like elementary particles. These models provided a convenient taxonomy for
the zoo of particles which had been identified in experiments, but it was
unclear whether the ``quarks'', to use Gell-Mann's term, represented actual
physical entities. Five years later, Feynman and Bjorken and Paschos postulated
that the quarks -- they called them partons -- would behave quasi-free at high
energies \cite{Feynman:1969ej, Bjorken:1969ja}. A consequence of this model is
that in the high energy limit the structure functions of the proton measured in
deep inelastic scattering depend only on the (dimensionless) ratio of the
momentum transfer of the virtual photon and the energy loss of the scattered
electron. This ``Bjorken scaling'' behavior was soon observed at SLAC by
Friedman, Kendall, and Taylor \cite{Breidenbach:1969kd}. Physicists were
initially reluctant to identify the partons implied by the SLAC experiment with
the quarks in the models by Gell-Mann and Zweig, but eventually it became clear
that they were one and the same.

In the simple parton model Bjorken's $F_1$ structure function is expressed in
terms of the number densities $q(x)$ of quarks and $\bar q(x)$ of antiquarks
as
%
\begin{equation}
  F_1(x, Q^2) = \frac{1}{2}\sum_{j}{e_j^2[q_j(x) + \bar{q}_j(x)]}
\end{equation}
%
where the sum is taken over quark flavors $j$ and $e_j$ is the electromagnetic
charge of flavor $j$. In longitudinally polarized DIS we define an analogous
polarization density $\Delta q(x) \equiv q_+(x) - q_-(x)$ as the difference in
number density between quarks whose spins are aligned with the (longitudinal)
spin of the proton and quarks whose spins are anti-aligned; the polarized
analogue to $F_1$ is then
%
\begin{equation}
  g_1(x, Q^2) = \frac{1}{2}\sum_{j}{e_j^2[\Delta q_j(x) + \Delta \bar{q}_j(x)]}.
  \label{eqn:simple-g1}
\end{equation}

Typically one assumes \(SU(3)_F\) flavor symmetry and thus it is useful to
express the integral of \(g_1\) in terms of quantities which have specific
\(SU(3)_F\) transformation properties:
%
\begin{equation}
  \Gamma_1^p = \int_0^1 dx~g_1^p(x) = \frac{1}{9}\left[a_0 + \frac{3}{4}a_3 + \frac{1}{4}a_8\right].
  \label{eqn:g1}
\end{equation}
%
The \(a_j\) are the hadronic matrix elements of an octet of quark $SU(3)_F$
axial-vector currents $J_{5\mu}^i$ and a flavor singlet axial current
$J_{5\mu}^0$, and are related to the polarized quark densities in the proton as
%
\begin{eqnarray}
  a_0 & = & (\Delta u + \Delta \bar{u}) + (\Delta d + \Delta \bar{d}) + (\Delta s + \Delta \bar{s}) \nonumber \\
  a_3 & = & (\Delta u + \Delta \bar{u}) - (\Delta d + \Delta \bar{d}) \nonumber \\
  a_8 & = & (\Delta u + \Delta \bar{u}) + (\Delta d + \Delta \bar{d}) - 2(\Delta s + \Delta \bar{s})
  \label{eqn:su3-dis}
\end{eqnarray}
%
In the limit of massless partons the non-singlet currents are scale-independent quantities, and are known from $\beta$-decay measurements
\cite{Amsler:2008zzb}:
% Stiegler says a_8 = 3F+D, but Anselmino and Ashman agree on the (-)
\begin{eqnarray}
  a_3 & = & g_A = F+D = 1.2670 \pm 0.0035 \nonumber \\
  a_8 & = & 3F-D = 0.585 \pm 0.025.
  \label{eqn:beta-decay}
\end{eqnarray}
%
Hence a measurement of \(\Gamma_1^p\) allows the extraction of the flavor
singlet \(a_0\), the quark spin contribution to the spin of the proton. If one
assumes that the strange quark distribution does not contribute to the spin of
the proton, as Ellis and Jaffe did in 1974 \cite{Ellis:1973kp}, Equations
\ref{eqn:su3-dis} and \ref{eqn:beta-decay} allow a \textit{prediction} of the
quark spin contribution to the spin of the proton, namely \(a_0 = a_8 \approx
0.6\).
