\section{Charged Pion Identification}

Charged pions are identified and separated from kaons, protons, and electrons by the amount of energy they lose in the TPC.  The dE/dx of a TPC track is obtained by sorting the track hits according to energy loss, removing the top 30\%, and averaging the rest.  Track dE/dx values for a given particle species at a fixed momentum are Gaussian, so one can also express the dE/dx value for each track in terms of a deviation from the mean dE/dx for some identified particle at that track's momentum.  In particular, track energy loss values at STAR are commonly given in terms of ``$n\sigma(\pi)$'', the deviation from the mean of the pion peak divided by the width of said peak.  Protons and kaons fall to the left of the pion peak (lower energy loss) and electrons fall to the right (higher energy loss).

The peak position of the raw $n\sigma(\pi)$ distribution generated by the STAR reconstruction software exhibits some significant time dependence, so instead of assuming a fixed mean of 0.0 for the pion Gaussian,  this analysis performs a triple Gaussian fit on the $n\sigma(\pi)$ distribution for each fill and extracts time-dependent means to better calibrate the PID cut.  After this recalibration one can extract yields for the various species of charged particles by fitting the $n\sigma(\pi)$ distributions with a multi-Gaussian parametric function.  The fitting procedure starts with 8 Gaussians -- one each for $\pi^{+}$, $\pi^{-}$, $K^{+}$, $K^{-}$, $p$, $\bar{p}$, $e^{+}$, and $e^{-}$.  The number of free parameters is reduced by applying the following constraints:

\begin{itemize}
    \item all widths must be equal (dE/dx resolution isn't particle-dependent)
    \item particle/antiparticle pairs should have the same mean
    \item $\pi - K$, $\pi - p$, and $\pi - e$ separations are known from other analyses \cite{Xu:2008th}
\end{itemize}

In the end there are 24 - 7 - 4 - 3 = 10 free parameters in the fit:  the Gaussian width, the position of the $\pi$ Gaussian, and the yields.  The particle separations change as a function of momentum, not $p_{T}$, so we slice a $p_{T}$ bin into momentum bins and fit each one individually.  Figure \ref{fig:typical-nsigmapi} shows a typical fit result.  The tracks have been shifted by 6*charge in order to plot positive and negatively charged tracks on the same histogram.

\begin{figure}
  \begin{center}
    \includegraphics[width=0.7\textwidth]{figures/typical-nsigmapi}    
  \end{center}
  \caption{Example PID fit result}
  \label{fig:typical-nsigmapi}
\end{figure}

With this database of particle yields in hand we can calculate the set of PID cuts that minimize the statistical uncertainty on the background-subtracted $A_{LL}$ (Equation \ref{eqn:sigma-all}).  A simple minimization routine assuming $\sigma_{A_{LL}}^{2} = 1/N$ for the raw asymmetries yields the results in Table \ref{tbl:pid-selection-windows}.

\begin{table}
    \begin{center}
        \begin{tabular}{c|ccc}
        \hline
        $p_{T}$ bin & $\pi$ window & proton/kaon max & electron min\\
        \hline
        \hline
        [2.00 - 3.18] & (-1.10, 2.30) & -2.10 & 2.60\\
        \hline
        [3.18 - 4.56] & (-1.40, 2.10) & -2.10 & 2.40\\
        \hline
        [4.56 - 6.32] & (-1.40, 1.80) & -2.10 & 2.40\\
        \hline
        [6.32 - 8.80] & (-1.40, 1.80) & -2.10 & 2.40\\
        \hline
        [8.80 - 12.84] & (-1.30, 1.40) & -2.10 & 2.10\\
    \hline
    \end{tabular}
    \end{center}
    \caption{PID Selection Windows}
    \label{tbl:pid-selection-windows}
\end{table}
