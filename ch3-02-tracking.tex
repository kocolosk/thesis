\section{Tracking}

Tracking, the procedure by which charged particle trajectories are
reconstructed from a set of detector responses, and vertexing, the
determination of the position along the beam line where particles from the two
beams participated in a hard scattering, are interdependent at STAR. Both rely
heavily on the TPC; in this analysis the TPC is the sole tracking detector,
and TPC tracks are the primary source of information used by the vertex
finder. The tracker \cite{Rose:2003wx} and vertex finder
\cite{vertex-finder-starnote} aspire to
%
\begin{itemize}
  \item find tracks based on a collection of hits from the tracking detectors,
  \item fit the hits associated to a track using a suitable track model,
  \item identify an event vertex and the subset of reconstructed tracks (called primary tracks) associated with that vertex,
  \item refit the primary tracks incorporating the event vertex, and
  \item calculate final state particle information for all reconstructed tracks.
\end{itemize}
%
The wide variety of collision types analyzed at STAR present a number of
challenges for the tracking and vertexing software. Head-on (central)
collisions of gold atoms can produce 6000 or more charged particles in a
single event, while high luminosity proton beams can generate up to $\sim$40
``pile-up'' collisions in the TPC volume during the 40$\mu$s drift time
required to read out a single triggered bunch crossing.

STAR employs a Kalman Filter approach to combine track finding and fitting
into a single process. The essence of the Kalman approach is that the current
properties of a track segment are used to define the search space for the next
hit to be added to the track, and when that next hit is added, the track
properties are incrementally refined incorporating the new information. The
tracker starts by identifying a track seed, a short sequence of only a few
hits containing the minimum amount of information needed to estimate the
track's direction and curvature. In principle these seeds could be constructed
using hits from any region of the detector, but the most effective approach is
to construct them using hits from the periphery of the TPC where the hit
concentration is lowest. The Kalman search proceeds to extrapolate this track
seed inward to the next active layer of the detector, in this case a TPC pad
row. Matching hits on this layer are sought within a radius determined by the
errors associated with the track extrapolation. If no matching hit is found,
the extrapolation moves on to the next layer. If one or more hits are found,
the filter calculates the increase in chi-square associated with the addition
of each hit to the existing track segment and chooses the hit with the lowest
incremental chi-square if it falls below a configurable threshold. Once a hit
is added, the track parameters are updated using the Kalman track model and
the search continues on the next detector layer. The search terminates when it
reaches the innermost layer of the detector or when it passes some number of
layers without finding any matching hits. At the culmination of the search the
track parameters calculated at the innermost hit offer the best description of
the track; the other nodes can be considered under-constrained because they do
not incorporate information from nodes that are closer to the origin. An
outward track refit is performed to rectify this situation. The refit updates
track parameters using the same methods as the initial search; the only
difference from the search is that the hits are known in advance.

On occasion a track seed is identified deeper in the TPC; in these cases, an
outward extrapolation of the track is warranted, and an outward search very
similar to the inward one is performed. Once this search is complete, the
inner portion of the track is refit to incorporate the information from the
new outer nodes.

The tracker continues the finding and fitting procedure with new track seeds
until no more seeds are found. The tracks identified at this stage of
reconstruction do not incorporate any information about the position of the
event vertex and are termed ``global tracks''. Some global tracks are pileup
tracks from events other than the one that triggered the detector, some
correspond to the products of strange decays in the detector, and some really
are ``primary tracks'' from particles produced in the hard scattering that
fired the event trigger(s). Determining which tracks belong in each category
requires a precise knowledge of the position of the event vertex, and it turns
out that the best calculation of the vertex position is one that relies on the
global tracks themselves.

% TODO transverse momentum determination needs to be described somewhere
