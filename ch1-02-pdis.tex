\section{First Experimental Tests}

In polarized deep inelastic scattering, a longitudinally polarized lepton beam is scattered off of nucleon targets polarized parallel or perpendicular to the beam axis. Asymmetries are formed by comparing event rates for scattering in different spin configurations.  For a spin $\frac{1}{2}$ target, the asymmetries of interest are

% Stiegler does not include the 1/2 nor the differential in this equation
\begin{equation}
  A_{\parallel} = \frac{d\sigma^{\rightarrow \Leftarrow} - d\sigma^{\rightarrow \Rightarrow}}{2d\sigma_{unpol}}, ~~~~~~~
  A_{\perp} = \frac{d\sigma^{\rightarrow \Uparrow} - d\sigma^{\rightarrow \Downarrow}}{2d\sigma_{unpol}}
\end{equation}

Spin-dependent cross sections can be calculated by contracting the elastic Compton amplitude $T_{\mu \nu}$ with the photon polarization vectors; in the presence of parity conservation and time reversal, four of these are independent \cite{??}:
%
\begin{eqnarray}
  \sigma_{1/2} & = & F_1 + g_1 - \gamma^2 g_2, \nonumber \\
  \sigma_{3/2} & = & F_1 - g_1 + \gamma^2 g_2, \nonumber \\
  \sigma_L & = & -F_1 + F_2(1+\gamma^2)/(2x),  \nonumber \\
  \sigma_{TL} & = & \sqrt{2}\gamma (g_1+g_2).
\end{eqnarray}
%
Here $\gamma^2 = Q^2/v^2$.  These four cross sections are commonly rearranged into a pair of virtual photon asymmetries $A_1$ and $A_2$:
%
\begin{equation}
  A_1 = \frac{\sigma_{1/2} - \sigma_{3/2}}{\sigma_{1/2} + \sigma_{3/2}}, ~~~~ A_2 = \frac{\sigma_{TL}}{\sigma_T}
\end{equation}
%
The longitudinal and transverse DIS asymmetries can then be written in terms of these virtual photon asymmetries.  In the case of $A_{\parallel}$ we have
\begin{equation}
  A_{\parallel} = D(A_1 + \eta A_2),
\end{equation}
%
where the coefficients $D$ and $\eta$ can be approximated to first order in $\gamma$ in terms of the usual DIS kinematic variables and $R = \frac{\sigma_{L}}{\sigma_T}$:
% include coeffs for A_perp here too?
% Detailed derivation of these results can be found in \cite{Anselmino:1994gn}
\begin{equation}
  D \approx \frac{y(2-y)}{y^2 + 2(1-y)(1+R)}, ~~~~~~~~ \eta \approx \frac{2(1-y)}{y(2-y)} \frac{\sqrt{Q^2}}{E}.
\end{equation}
%
(Similar equations exist for $A_{\perp}$, such that a measurement of both asymmetries allows an extraction of both $A_1$ and $A_2$). $D$ can be thought of as a depolarization factor arising from the fact that the photon is not fully aligned with the lepton beam, and $\eta$ is a kinematic factor that is usually small.  Finally, the polarized structure functions can be written in terms of $A_{1,2}$:
\begin{equation}
  g_1 = \frac{F_2}{2x(1+R)}(A_1+\gamma A_2), ~~~~~ g_2 = \frac{F_2}{2x(1+R)}(A_2/\gamma - A_1).
\end{equation}
Thus, measurements of $A_{\parallel}$, $A_{\perp}$, $F_2$, and $R$ are sufficient to extract the polarized structure functions of the nucleon.

% note that they just measured A_parallel and placed a limit on the A_2 contribution
The first DIS experiments to extract $g_1$ using this methodology were E80 and E130, conducted in the late 1970s and early 1980s at SLAC.  These experiments scattered longitudinally polarized electron beams off of longitudinally polarized proton targets and were able to measure $A_{\parallel}^p$ in the range $0.1 < x < 0.7$.  Using the positivity limit $A_2 < \sqrt{R}$ they determined that $A_{\parallel}/D$ was a good approximation for $A_1$, and after exploiting that assumption their results were consistent with expectations from the parton model \cite{Alguard:1976bm, Baum:1983ha}. % there are 2-3 more result papers cited in Baum:1983ha if I want them

In 1988, the European Muon Collaboration (EMC) published data on asymmetries of longitudinally polarized muon beams scattering off of longitudinally polarized proton targets.  The EMC experiment boasted kinematic coverage down to $x = 0.01$, an order of magnitude lower than the earlier SLAC experiments, and they extracted measurements of the proton's $g_1$ structure function using the same assumption that $A_1 \approx A_{\parallel}/D$. The EMC data on $A_1$ were consistent with the results from SLAC in their overlapping kinematic regime, but at low $x$ the EMC results deviated significantly from parton model predictions. As shown in Figure \ref{fig:emc-g1p}, the integral value of $g_1^p$ obtained from the EMC extraction was incompatible with the prediction from Ellis and Jaffe.  Applying the Bjorken sum rule to obtain a (significantly negative) value for the integral of $g_1^n$, EMC quoted the following quark spin contribution to the spin of the proton \cite{Ashman:1987hv}: % Ashman:1989ig might be the authoritative source
%
\begin{equation}
  \langle S_z \rangle_{u+d} = 0.068 \pm 0.047 \pm 0.103.
\end{equation}
%
This result sparked what was once termed a ``spin crisis'' in particle physics.  Sucessive polarized DIS experiments at CERN, SLAC, and DESY confirmed and refined the EMC measurement of $g_1^p$ with improved precision over a wider kinematic range \cite{Adams:1994zd}, and measured both $g_1^n$ \cite{Anthony:1993uf} and $g_1^d$ \cite{Adeva:1993km} which allowed a verification of the critical Bjorken sum rule.

\begin{figure}
  \includegraphics[width=1.0\textwidth]{figures/emc-g1p}
  \caption{EMC extraction of $g^1_p$ and its integral compared to the prediction from Ellis-Jaffe \cite{Ashman:1987hv}}
  \label{fig:emc-g1p}
\end{figure}

  Even in that context, the EMC result was surprising.  Applying the Bjorken sum rule to obtain $\int g_1^n$ and again assuming an unpolarized sea, one can solve for the individual quark spin contributions to the spin of the proton:

\begin{equation}
  \int_0^1 dx~g_1^p(x,Q^2) - g_1^n(x,Q^2) = \frac{g_A}{6}(1 + O(\alpha_s))
\end{equation}

% Rearranging \ref{eqn:g1} to solve for $\Delta \Sigma$ and plugging in values from \ref{eqn:beta-decay} and EMC one finds
% 
% \begin{equation}
%   \Delta \Sigma = 9\Gamma_1^p - \frac{3}{4}a_3 - \frac{1}{4}a_8 = 
% \end{equation}
% 
%   The data covered a sufficient range
% EMC measures first moment of g1, taking a3 and a8 from beta-decay measurements means that we can extract a0 from $\Gamma_1^p$ and it's $\sim$ 0.  But
% 
% \begin{equation}
%   a_0 = \Delta \Sigma = a_8 + 3(\Delta s + \Delta \bar{s})
% \end{equation}
% 
% from above, and if you ignore strange quark contributions (Ellis-Jaffe) you get $a_0 \sim 0.59$, obviously in stark contrast to EMC.  This is the original ``spin crisis''.  And of course $a_0 = 2<S_z^{quarks}>$.
% 
% Any need to mention ``Cloudy Bag'' model here?  I think not.
% 
% 
% \begin{equation}
%   y \equiv \frac{\nu}{E} = \frac{P \cdot q}{P \cdot k}
% \end{equation}
% 
% \begin{equation}
%   \gamma^2 = \frac{4M^2x^2}{Q^2}
% \end{equation}
