% discuss asympt. freedom, factorization, pdfs in "QCD and improved model"

% Then, in Access to DeltaG section, talk about scaling violation results and then pp asymmetries.

\section{QCD and the improved Parton Model}

The parton model results presented thus far assume that the photon scatters off
a free quark, a simplification which completely neglects the strong interaction.
In reality, quarks in the proton are tightly bound, constantly radiating and
reabsorbing gluons. Quantum chromodynamics (QCD) enhances the parton model with
interaction-dependent modifications of the simple parton model formulae. These
improvements do not alter the fundamental conclusions regarding the spin
composition of the proton established using the parton model, but they are
useful for a proper discussion of current experimental efforts to understand the
polarized gluon distribution.

\subsection{Factorization and Scaling Violations}

\begin{figure}
  \includegraphics[width=1.0\textwidth]{figures/radiative-corrections}
  \caption{Examples of first-order radiative corrections to the quark-photon vertex.}
  \label{fig:radiative-corrections}
\end{figure}

Figure \ref{fig:radiative-corrections} illustrates two first-order gluonic
corrections to the quark-photon vertex. The diagrams contain collinear
divergences due to the massless quarks, so the size of each correction is
actually infinite. The standard technique for dealing with these infinities is
to factorize the reaction into a hard part calculable in perturbative QCD and a
soft part which must be parameterized from experimental observations. Generally
speaking, one encounters terms of the form
%
\begin{equation}
  \alpha_s~ln \frac{Q^2}{m_q^2} = \alpha_s~ln \frac{Q^2}{\mu^2} + \alpha_s~ln \frac{\mu^2}{m_q^2}.
\end{equation}
%
The first term on the right is included in the hard part of the interaction and
the second (infinite) term is absorbed into the parton distribution functions.
The factorization scale \(\mu^2\) is an arbitrary number, and exact physical
results cannot depend on it, but since perturbative results are never exact the
particular choice of scale is important. A common choice is
\(\mu^2 = Q^2\), which means that perfect Bjorken scaling of the distribution
functions is broken; that is \(q(x) \rightarrow q(x,Q^2)\). However, the
dependence on \(Q^2\) is only logarithmic and is calculable by a set of
evolution equations which are presented later.

QCD factorization is completely general and not limited only to higher-order
corrections to DIS. Consider the case of mid-rapidity pion production at a high
energy proton-proton collider. The factorization theorem allows one to write the
hadronic cross section for this process as a convolution of several independent
entities: PDFs describing the probability of picking out a given parton from
each proton, a hard partonic cross section, and a fragmentation function giving
the probability that an outgoing quark will fragment into a pion with a fraction
\(z\) of the parton's momentum. To wit:
%
\begin{equation}
  \sigma_{p_A+p_B \rightarrow \pi+X} = \sum_{a,b,c} f_a(x_A, Q^2) \otimes f_b(x_B, Q^2) \otimes \sigma_{a+b \rightarrow c + X} \otimes D_c^{\pi}(z)
  \label{eqn:factorization}
\end{equation}
%
The sum is over all parton flavors that contribute to the hadronic cross
section. The \(f_i\) are the parton distribution functions; \(D_c^{\pi}(z)\) is
the pion fragmentation function for quark flavor \(c\). The partonic cross
section is calculable using perturbative QCD when the \(Q^2\) of the interaction
is large, while the PDFs and FFs must be parameterized from experimental
results. Fortunately, those distribution functions are \textit{universal}; they
can be measured in any process (commonly \(e^+e^-\) collisions) and then applied
in the calculation of any other cross section. 

\subsection{Spin Sum Rules}

One can write a classic sum rule for the proton spin which takes the form
\cite{Hagler:1998kg, Harindranath:1998ve, Bashinsky:1998if}
%
\begin{equation}
  \frac{1}{2} = \frac{1}{2}\Delta \Sigma_q + \Delta G + L_q + L_g.
  \label{eqn:jaffe-sum-rule}
\end{equation}
%
Unfortunately, the last two terms are not experimentally accessible. Lattice QCD
can measure angular momentum contributions to the proton spin, but the angular
momentum operators in the sum rule are non-local in a generic gauge. It is only
in the light-cone gauge, inaccessible to the lattice, where the terms in the sum
rule can be cleanly interpreted as individual parton helicity and angular
momentum contributions.

There exists an alternative sum rule \cite{Jaffe1990509, Ji:1996ek}
%
\begin{equation}
  \frac{1}{2} = \frac{1}{2}\Delta \Sigma_q + \hat{L}_q + \hat{J}_g
\end{equation}
%
where \(\hat{L}_q\) and \(\hat{J}_g\) correspond roughly to the orbital angular momentum of quarks and the total angular momentum of gluons in the nucleon. \(\hat{J}_q = \frac{1}{2}\Delta \Sigma + \hat{L}_q\) is measurable using deeply virtual Compton scattering, and \(\hat{J}_g\) can then in principle be obtained through the evolution of \(\hat{J}_q\).

The subtleties involved in decomposing the spin of the proton into experimental observables should not be viewed as a deterrent towards further experiments.  On the contrary, they are an opportunity for the data to lead the way to greater insight.
