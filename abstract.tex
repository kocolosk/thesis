Spin dependent measurements provide incisive tests for modern theories of particle physics.  The discovery that the integral of the quark helicity distribution is much too small to account for the spin of the proton is an excellent case in point.  It inspired an intense period of theoretical scrutiny and a new generation of experiments to study \(\Delta g\), the helicity distribution of gluons in the nucleon.  In particular, the unique polarized proton collider at RHIC enables a class of asymmetry measurements that are directly sensitive to gluon polarization.

The STAR experiment at RHIC measures the double spin asymmetry \(A_{LL}\) for a variety of final states in collisions of longitudinally polarized protons in order to constrain \(\Delta g\). Asymmetries for mid-rapidity charged pion production benefit from large cross sections and the excellent tracking and particle identification capabilities of the STAR Time Projection Chamber. This thesis presents the first measurements of charged pion \(A_{LL}\) at STAR. The measurements are compared to predictions based on perturbative QCD calculations and from this comparison model-dependent constraints are placed on the integral gluon polarization \(\Delta G\).
